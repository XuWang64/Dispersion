\documentclass[aps,tightenlines,16pt]{ctexart}

\usepackage{slashed}
\usepackage{amsmath,amsfonts,amssymb}
% \usepackage{bm}  %黑体希腊字母
\usepackage{bbm} %空心字母数字
\usepackage{ctex}
\usepackage{float}
%\usepackage[dvips]{graphicx}
\usepackage[english]{babel}
\allowdisplaybreaks[4]  %公式环境换行
\numberwithin{equation}{section}
\usepackage{slashed} %费曼slashed
\usepackage[left=2cm,right=2cm,top=2.54cm,bottom=2.54cm]{geometry}
%\usepackage[dvipdfm,pdfstartview=FitH]{hyperref}
%\usepackage{cancel}
%\usepackage{forest}
%\usepackage{leftidx}  %设置上下标
\usepackage{cite}
\usepackage[justification=centering]{caption}
\usepackage{graphicx, subfigure}
\usepackage{indentfirst}  %段落首行缩进
\usepackage{hyperref}  %设定引用公式跳转链接
\usepackage{color}  %设置字体颜色
\usepackage{tikz,pgf}
%\usepackage{tikz-feynman} 
%\tikzfeynmanset{compat=1.1.0}
\usepackage{braket}
%\usepackage{txfonts}  %平行符号
%\usepackage{fancyhdr}  %左偶右奇
\usepackage{multirow}
\usepackage{booktabs}
\usepackage{cancel}   
\usepackage{threeparttable}  
\usepackage{diagbox}
\usepackage{extpfeil}  %等号上加文字
\usepackage{extarrows}

\usetikzlibrary{calc}
%\usetikzlibrary{arrows.meta}
\usetikzlibrary{intersections}
\usetikzlibrary{trees}
\usetikzlibrary{decorations.pathmorphing}
\usetikzlibrary{decorations.markings}
\usetikzlibrary{patterns}
\tikzset{
   global scale/.style={
      scale=#1,
      every node/.append style={scale=#1}},
   photon/.style={decorate, decoration={snake}, draw=red},
   nucleon/.style={draw=black, postaction={decorate},
      decoration={markings,mark=at position .55 with{\arrow[draw=black]{>}}}},
   pion/.style={draw=blue, postaction={decorate},
      decoration={markings,mark=at position .55 with{\arrow[draw=blue]{}}}},
    }


% \newcommand{\bl}{\boldsymbol{l}}
% \newcommand{\bk}{\boldsymbol{k}}
% \newcommand{\bp}{\boldsymbol{p}}
% \newcommand{\bP}{\boldsymbol{P}}
% \newcommand{\bq}{\boldsymbol{q}}
% \newcommand{\bA}{\boldsymbol{A}}
% \newcommand{\bM}{\boldsymbol{M}}
% \newcommand{\bV}{\boldsymbol{V}}
% \newcommand{\ba}{\boldsymbol{a}}
% \newcommand{\bb}{\boldsymbol{b}}
% \newcommand{\bx}{\boldsymbol{x}}
% \newcommand{\bep}{\boldsymbol{\epsilon}}
% \newcommand{\bsi}{\boldsymbol{\sigma}}
% \newcommand{\bL}{\boldsymbol{L}}
% \newcommand{\bJ}{\boldsymbol{J}}
% \newcommand{\br}{\boldsymbol{r}}
% \newcommand{\bs}{\boldsymbol{s}}
% \newcommand{\bS}{\boldsymbol{S}}
% \newcommand{\bi}{\boldsymbol{i}}
% \newcommand{\bI}{\boldsymbol{I}}
% \newcommand{\bB}{\boldsymbol{B}}  
% \newcommand{\sP}{\slashed{P}} 
% \newcommand{\spp}{\slashed{p}} 
% \newcommand{\sk}{\slashed{k}} 
% \newcommand{\sq}{\slashed{q}}
% \newcommand{\sD}{\slashed{D}} 
% \newcommand{\sA}{\slashed{A}}
% \newcommand{\sep}{\slashed{\epsilon}} 
% \newcommand{\spar}{\slashed{\partial}} 
% \newcommand{\Pmu}{P^\mu} 
% \newcommand{\pmu}{p^\mu} 
% \newcommand{\kmu}{k^\mu} 
% \newcommand{\qmu}{q^\mu}
% \newcommand{\gmu}{\gamma^\mu}
% \newcommand{\bpi}{\boldsymbol{\pi}}
% \newcommand{\btau}{\boldsymbol{\tau}}
% \newcommand{\brho}{\boldsymbol{\rho}}
% \newcommand{\md}{\mathrm{d}}
% \newcommand{\mB}{\mathbf{B}}
% \newcommand{\mO}{\mathcal{O}}
\newcommand{\mL}{\mathcal{L}}
% \newcommand{\bm}[1]{\mbox{\boldmath{$#1$}}}

\allowdisplaybreaks


\begin{document}\large
     %\title{色散关系阅读笔记} 
     \title{色散关系}
     
\renewcommand{\today}{\number\year 年 \number\month 月 \number\day 日}
 \author{王旭}
 \maketitle
 %\newpage
 \setlength{\parindent}{2em}  %首行缩进两个中文字符
 \hypersetup{hypertex=true,
            colorlinks=true,
            linkcolor=blue,
            anchorcolor=blue,
            citecolor=blue}  %设定引用公式跳转链接
 \renewcommand\thesubsection{\arabic {subsection}}
 \renewcommand\contentsname{目录}
\tableofcontents
\newpage 
本笔记基本参考\cite{oller2019brief}。
\section{S矩阵及分波}
用$| \mathbf{q},\sigma,m,s,\lambda\rangle$表示一个粒子态,粒子态的归一化如下
\begin{align}
   \langle \mathbf{q},\sigma,m,s,\lambda||\mathbf{q}^{\prime},\sigma^{\prime},m^{\prime},s^{\prime},\lambda^{\prime}\rangle=(2\pi)^3 2q^0 \delta^3(\mathbf{q}-\mathbf{q}^{\prime})\delta^{\sigma\sigma^{\prime}}\delta^{ss^{\prime}}\delta^{\lambda\lambda^{\prime}},
\end{align}
\subsection{S矩阵}
在QFT中,从初态$|i\rangle$到末态的$|f\rangle$的几率由幺正矩阵S描述,S矩阵元定义为
\begin{align}
   S_{fi}=\langle f|S|i\rangle.
\end{align}
在相互作用表象中,可表示为
\begin{align}
   S_{fi}=\frac{\langle f |e^{\mbox{i}\int d^4x \mL_{int}} |i\rangle}{\langle 0|e^{\mbox{i}\int d^4 x \mL_{int}}|0\rangle}.
\end{align}

利用S矩阵定义T矩阵为
\begin{align}
   S=1+\mbox{i}T,
\end{align}
则由S矩阵的幺正性可得
\begin{align}
   T-T^{\dagger}=\mbox{i}TT^{\dagger},
\end{align}
两边夹上$\langle f|$和$|i\rangle$,以及在右手边插入单位矩阵,则可以得到
\begin{align}
   \begin{aligned}
   \langle f|T|i\rangle - \langle f|T^{\dagger}|i\rangle  =
      &\mbox{i}\sum \int \Big[
      (2\pi)^4\delta(p_f-\sum_{i=1}^{n}q_i) \prod_{i=1}^n\frac{d^3q_i}{(2\pi)^3 2q_i^0}\Big]\\
      &\times \langle f|T|\mathbf{q}_1,\sigma_1,m_1,s_1,\lambda_1;\cdots;\mathbf{q}_n,\sigma_n,m_n,s_n,\lambda_n\rangle\\
      &\times\langle \mathbf{q}_1,\sigma_1,m_1,s_1,\lambda_1;\cdots;\mathbf{q}_n,\sigma_n,m_n,s_n,\lambda_n|T|i\rangle,
   \end{aligned}
\end{align}
用$\int dQ$表示末态相空间积分
\begin{align}
   \int dQ=\int (2\pi)^4\delta(p_f-\sum_{i=1}^{n}q_i)\prod_{i=1}^{n}\frac{d^3q_i}{(2\pi)^3 2q_i^0},
\end{align}
两体末态相空间在质心系中,
\begin{align}
   dQ=\int\frac{d^3p_1}{(2\pi)^3 2p_1^0}\frac{d^3p_2}{(2\pi)^3 2p_2^0}(2\pi)^4\delta(p_1+p_2)=\frac{|\mathbf{p}_1|d\Omega}{16\pi^2\sqrt{s}},
\end{align}
在两体散射中,选取质心参考系,则散射截面为
\begin{align}
   \sigma_{fi}=\frac{1}{4|\mathbf{p}_1\sqrt{s}}\int dQ_f |\langle f|T|\mathbf{p}_1,\sigma_1,m_1,s_1,\lambda_1;|\mathbf{p}_2,\sigma_2,m_2,s_2,\lambda_2\rangle|^2,
\end{align}
对于给定初态$|i\rangle$,则所有可能末态的散射截面为
\begin{align}
   \sigma_{i}=\frac{1}{4|\mathbf{p}_1|\sqrt{s}}\sum_{f}\int dQ_f |\langle f|T||\mathbf{p}_1,\sigma_1,m_1,s_1,\lambda_1;|\mathbf{p}_2,\sigma_2,m_2,s_2,\lambda_2\rangle|^2,
\end{align}
选取初态$|i\rangle$等于末态$|f\rangle$,则可以得到光学定理
\begin{align}
   \Im T_{ii}=\frac{1}{2}\sum_f\int dQ_f|T_{fi}|^2=2|\mathbf{p}_1|\sqrt{s}\sigma_i.
\end{align}

\subsection{分波}
用$|\mathbf{p},\sigma_1\sigma_2\rangle$一个两粒子态,其中$\mathbf{p}$表示质心系中三动量,各自静止系中自旋第三分量表示为$\sigma_1,\sigma_2$的粒子,定义一个两体态$|lm,\sigma_1\sigma_2\rangle$
\begin{align}
   |lm,\sigma_1\sigma_2\rangle = \frac{1}{\sqrt{4\pi}}\int d \hat{\mathbf{p}} Y_l^m(\hat{\mathbf{p}})|\hat{\mathbf{p}},\sigma_1\sigma_2\rangle,
\end{align}
其中$l$为轨道角动量,$m$为其第三分量。

旋转算符$R$作用在态$|\mathbf{p},\sigma_1\sigma_2\rangle$,结果为
\begin{align}
   R|\mathbf{p},\sigma_1 \sigma_2 \rangle = \sum_{\sigma_1^{\prime},\sigma_2^{\prime}} D^{(s_1)}(R)_{\sigma_1^{\prime}\sigma_1} D^{(s_2)}(R)_{\sigma_2^{\prime}\sigma_2} |\mathbf{p}^{\prime},\sigma_1^{\prime}\sigma_2^{\prime}\rangle,
\end{align}
则$R$作用在$|lm,\sigma_1\sigma_2\rangle$上,结果为
\begin{align}
   \begin{aligned}
   R|lm,\sigma_1\sigma_2\rangle = &\sum_{\sigma_1^{\prime},\sigma_2^{\prime}} D^{(s_1)}(R)_{\sigma_1^{\prime}\sigma_1} D^{(s_2)}(R)_{\sigma_2^{\prime}\sigma_2} \frac{1}{\sqrt{4\pi}}\int d \hat{\mathbf{p}^{\prime}} Y_l^m(R^{-1}\hat{\mathbf{p}}^{\prime})|\hat{\mathbf{p}}^{\prime},\sigma_1^{\prime}\sigma_2^{\prime}\rangle\\
   =&\sum_{\sigma_1^{\prime},\sigma_2^{\prime},m^{\prime}} D^{(l)}(R)_{mm^{\prime}} D^{(s_1)}(R)_{\sigma_1^{\prime}\sigma_1} D^{(s_2)}(R)_{\sigma_2^{\prime}\sigma_2} |lm^{\prime},\sigma_1^{\prime}\sigma_2^{\prime}\rangle.
   \end{aligned}
\end{align}

首先考虑两粒子自旋耦合,再考虑总自旋和轨道角动量的耦合,利用CG系数将$|lm,\sigma_1\sigma_2\rangle$转换到$|J\mu,lS\rangle$基底,有
\begin{align}
   |J\mu,lS\rangle=\sum_{\sigma_1,\sigma_2,m,M}(\sigma_1\sigma_2M|s_1s_2S)(mM\mu|lSJ)|lm,\sigma_1\sigma_2\rangle,
\end{align}
其中$s_1,s_2$表示两粒子的自旋,$\sigma_1,\sigma_2$分别表示自旋的第三分量,$S$表示两粒子的总自旋,$M$表示总自旋的第三分量,$J$为总角动量,$\mu$为总角动量的第三分量。

当考虑同位旋结构时,转换关系为
\begin{align}
   |J\mu,lS,It_3\rangle=\sum_{\sigma_1,\sigma_2,m,M,\alpha_1,\alpha_2}(\sigma_1\sigma_2M|s_1s_2S)(mM\mu|lSJ)(\alpha_1\alpha_2t_3|\tau_1\tau_2I)|\mathbf{p},\sigma_1\sigma_2,\alpha_1\alpha_2\rangle,
\end{align}
\section{Kallen-Lehmann表示}
\section{交叉对称及同位旋结构}
考虑$\pi\pi$

    
\newpage 

\renewcommand\refname{参考文献}

\bibliographystyle{unsrt}

\bibliography{Disp}

\end{document}
